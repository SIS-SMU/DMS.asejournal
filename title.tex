Diversity Maximization Speedup for Fault Localization
%\thanks{To make the paper {\it appear} more novel, instead of being incremental from "diagnostic prioritization", it may be better to set some distance between their approaches and ours. So, how about avoiding saying that ours is diagnostic prioritization? Then, we need to modify the abstract accordingly as well. David: Personally, I feel mentioning diagnostic prioritization is better, for the following reasons: diagnostic prioritization is not our contribution and we shouldn't claim it as such and spend much space to argue why it is important. We could just mention that it is an existing research direction evident from the many papers on this topic in good/top conferences and do need to argue why it is interesting. We could then focus on our actual contribution -- namely on approach to make diagnostic prioritization better. Also if change the title to test prioritization, people may view our approach as just one of the many existing test prioritization techniques which have been investigated for the past decade and thus our contribution might be viewed as marginal. Furthermore, we do not have much space to describe the importance of diagnostic prioritization ... :-) LX: agree on the point that we shouldn't mention test prioritization either; we should have a name of our own and use it in the title...On the other hand, I don't think we shall classify ourselves as "diagnostic prioritization" at the beginning because Liang developed his approaches (mostly) independent from diagnostic prioritization, even though we share similar goals with it...we need to explain the motivation for DMS anyway; we should just explain it in a way that is independent from diagnostic prioritization (DP); of course, we still compare the results with those DP techniques.}
%Diagnostic Prioritization Using Expected Merit Maximization
%Test Prioritization for Software Fault Localization based on Risk Analysis 