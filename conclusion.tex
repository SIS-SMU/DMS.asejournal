This paper proposes a new concept and technique named {\em Diversity Maximization Speedup} ({\sc Dms}) aiming to minimize the amount of effort in manual oracle construction, while still permitting effective fault localization. In comparison with six other existing prioritization techniques on 12 {\em C} programs, we have shown that our technique requires on average a smaller number of labelled test cases to achieve the targeted diagnostic cost of subsequent fault localization techniques, and that if the same number of labelled test cases are allowed, it can choose test cases that may be more effective in reducing debugging cost. We have shown that the improvements made by our technique on real-life programs over other existing techniques are statistically significant.

In future, we will evaluate the proposed approach on more subject programs. We will also explore the possibility of adopting more sophisticated trend analysis methods.

%Future work includes evaluation on more subject programs and further extension on our prioritization approach by adopting more sophisticated trend analysis methods.

%\vspace{-4pt}
\section{Acknowledgement} \label{sec.acknowledgement}
%\smallskip\noindent{\bf Acknowledgement.}
This work is partially supported by NSFC Program (No.61073006 and 61103032), Tsinghua University project 2010THZ0, and National Key Technology R\&D Program of the Ministry of Science and Technology of China (No2013BAH01B03). We thank researchers at University of Nebraska--Lincoln, Georgia Tech, and Siemens Corporate Research for the Software-artifact Infrastructure Repository. We would also like to thank the anonymous reviewers for providing us with constructive comments and suggestions.
