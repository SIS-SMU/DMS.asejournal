This paper proposes a new concept and technique aiming to minimize the amount of effort in manual oracle construction, while still permitting effective fault localization. In comparison with existing prioritization techniques on 12 {\em C} programs, we have shown that our technique only requires on average a small number of test cases to accomplish the target average cost within 1\% accuracy lost, and that it outperforms existing techniques in terms of reducing debugging cost for the subject programs. We have also shown that the differences on real-life programs are statistically significant.

In future, we will evaluate the proposed approach on more subject programs. We will also explore the possibility of adopting more sophisticated trend analysis methods.

%Future work includes evaluation on more subject programs and further extension on our prioritization approach by adopting more sophisticated trend analysis methods.

%\vspace{-4pt}
\section{Acknowledgement} \label{sec.acknowledgement}
%\smallskip\noindent{\bf Acknowledgement.}
This work is partially supported by NSFC Program (No.61073006 and 61103032), Tsinghua University project 2010THZ0, and National Key Technology R\&D Program of the Ministry of Science and Technology of China (No2013BAH01B03). We thank researchers at University of Nebraska--Lincoln, Georgia Tech, and Siemens Corporate Research for the Software-artifact Infrastructure Repository. We would also like to thank the anonymous reviewers for providing us with constructive comments and suggestions.
